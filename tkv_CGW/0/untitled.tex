\documentclass[a4paper,9pt,russian]{article}
\usepackage{cmap}
\usepackage{hyperref}
\usepackage[T1]{fontenc}
\usepackage[utf8]{inputenc}
\usepackage[russian]{babel}
\usepackage{graphicx}
\usepackage{xcolor}
\usepackage{amssymb}
\usepackage{amsmath}
\usepackage{physics}

\title{PГР 0}
\author{А. В. Козлов} 
\begin{document}
\maketitle
\begin{abstract}
В данной РГР нужно вспомнить прошлый семестр.
\end{abstract}
\section{Гармонический осциллятор}
\begin{itemize}
	\item Уравнение самого простого осциллятора без трения:
\[
	\ddot x + \omega_0^2 x = 0
.\]
	\item Гармонический осциллятор с трением:
	\[
	\ddot x +2\gamma\dot x+  \omega_0^2 x = 0
	.\] 
	\item Гармонический осциллятор с трением под действием произвольной силы:
		\[
		\ddot x +2\gamma\dot x+  \omega_0^2 x = F\qty(t)
		.\] 
	\item Гармонический осциллятор под действием гармонической силы:
		\[
			\ddot x +2\gamma\dot x+  \omega_0^2 x = F\sin{\omega t}
		.\] 
\end{itemize}

\end{document} 
