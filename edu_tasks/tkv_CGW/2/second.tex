\documentclass[a4paper,russian]{article}
\usepackage{cmap}
\usepackage{hyperref}
\usepackage[T2A]{fontenc}
\usepackage[utf8]{inputenc}
\usepackage[russian]{babel}
\usepackage{graphicx}
\usepackage{xcolor}
\usepackage{amssymb}
\usepackage{amsmath}
\usepackage{physics}

\renewcommand{\kappa}{\varkappa}
\renewcommand{\phi}{\varphi}
\renewcommand{\epsilon}{\varepsilon}

\title{РГР 2}
\author{А. В. Козлов} 
\begin{document}
\maketitle
\section{Формулировка}
Дана следующая бесконечная цепочка связанных пружинками 
грузиков (см. ...).
Требуется найти закон движения первого грузика: 
$\delta x_0\qty(t) - ?$ 

\section{Решение}
Запишем второй закон Ньютона для грузиков:
\begin{align}
 & m\ddot{\delta x_n} = k \qty(\delta x_{n-1} - 2\delta x_n +
 \delta x_{n+1}),\quad n\ne 0,1;\\
 & M\ddot{\delta x_n} = k \qty(\delta x_{n-1} - 2\delta x_n +
 \delta x_{n+1}) + \delta_{0,n}F\sin{\omega_0t},\quad n= 0,1.
\end{align}
Решения будем искать в виде:
\[
\delta x_n = 
\begin{cases}
	c_- e^{i\omega_0t-i\kappa_-n}+c.c. \quad n\le -1;\\
	C e^{i\omega_0t-i\kappa n}+c.c. \quad n = 0,1;\\
	c_+ e^{i\omega_0t-i\kappa_+ n}+c.c. \quad n\ge 2.
\end{cases}
\]
Если подставить такой вид решения в уравнения Ньютона, то для
$n\le -2$ и для $n\ge 3$ получаем:
\begin{gather}
	\omega_0^2=\dfrac{4k}{m}\sin^2{\dfrac{\kappa_-}{2}},
	\quad n \le -2;\\
	\omega_0^2=\dfrac{4k}{m}\sin^2{\dfrac{\kappa_+}{2}},
	\quad n\ge 3.
\end{gather}
Если учесть тот факт, что решение для $n \in (-\infty,-2] 
\cup [3,+\infty)$ должно иметь вид, как и в случае для
обычной бесконечной цепочки осцилляторов. Посему должно 
выполняться дисперсионное соотношение для капп:
\[
	\omega_0^2=4
	\dfrac{k}{m}\sin^2{\dfrac{\kappa_{\pm}}{2}}
.\]
Откуда получаем выражение для капп:
\[
	\kappa_\pm = \pm 2 \arcsin\qty(\frac{\omega_0\sqrt{\dfrac{m}{k}}}{2})
.\] 
Кроме того, принимаем во внимание, что решение должно
удовлетворять принципу причинности. То есть $n\kappa_{\pm}<0$.
Таким образом приходим к тому, что:
\begin{equation*}
\kappa_+ \ge 0, \quad \kappa_- = -\kappa_+ .
\end{equation*}
Тогда имеем:
\begin{equation}\label{k}
	\kappa_+ = - \kappa_-= \abs{2 \arcsin\qty(\frac{\omega_0\sqrt{\dfrac{m}{k}}}{2})}
\end{equation}
\par
Подставляем решения в уравнения Ньютона для прочих $n$ (см.
в тетради):
\begin{align}
 & n=-1:\quad -m\omega_0^2=k\qty(e^{i\kappa_-}-2+ 
 \dfrac{C}{c_-}e^{-i\kappa_-});\label{-1}\\
 & n=0:\quad -M\omega_0^2=k\qty(\dfrac{c_-}{C}e^{i\kappa_-}-2 
 +e^{-i\kappa})+\dfrac{F}{2iC};\label{0}\\
 & n=1:\quad -M\omega_0^2=k\qty(e^{i\kappa}-2+\dfrac{c_+}{C}
 e^{-2i\kappa_+ + i\kappa});\label{1}\\
 & n=2:\quad -m\omega_0^2=k\qty(\dfrac{C}{c_+}e^{2i\kappa_+-i\kappa} 
 -2+e^{-i\kappa_+}).\label{2}
\end{align}
Теперь поочерёдно рассмотрим различные пары данных выражений
и вынесем некоторые полезные для решения данные. Итак,
рассмотрим (\ref{-1},\ref{0}):
\[
\dfrac{k}{m}\qty(e^{-i\kappa_+}-2+\dfrac{C}{c_-}e^{i\kappa_+})=
\dfrac{k}{M}\qty(\dfrac{c_-}{C}e^{-i\kappa_+}-2+e^{-i\kappa})+\dfrac{F}{2iC}
,\]
откуда следует:
\begin{equation}\label{c-}
\dfrac{c_-}{C}=\dfrac{M}{m}.
\end{equation}
Теперь обращаем свой взор на (\ref{-1}, \ref{2}):
\[
\dfrac{m}{M}e^{i\kappa_+}=\dfrac{C}{c_+}e^{2i\kappa_+ - i\kappa}
,\] откуда выражаем отношение амплитуд:
\begin{equation}\label{c+}
\dfrac{c_+}{C}=\dfrac{M}{m}e^{i\kappa_+ -i\kappa}
.\end{equation}

В результате рассмотрения пары уравнений (\ref{0}, \ref{1}) можно получить соотношение:
\begin{equation}\label{kc}
	\sin{\kappa}=-\dfrac{F}{4kC}
.\end{equation}

Чтобы получить уравнение колебаний нулевого грузика 
нужно найти связь между $\kappa$ и  $\kappa_+$, например. 
Эту связь можно вытащить из (\ref{1}, \ref{2}):
\[
\dfrac{k}{M}\qty(e^{i\kappa}-2)=\dfrac{k}{m}\qty(\dfrac{m}{M}e^{i\kappa_+}-2)
,\]
тогда связь имеет вид:
\begin{equation}\label{k+}
e^{i\kappa}=e^{i\kappa_+}+2\qty(1-\dfrac{M}{m})
.\end{equation}

Если подставить (\ref{kc}) в (\ref{k+}) и выделить мнимую часть, то получится:
\begin{equation*}
	\sin{\kappa}=\omega_0\sqrt{\dfrac{m}{k}}\sqrt{1-{\omega_0^2 \dfrac{m}{k}}/4}=-\dfrac{F}{4kC}
,\end{equation*}
\begin{equation}\label{C_fin}
	C=-\dfrac{F}{2k\omega_0\sqrt{\dfrac{m}{k}}\sqrt{4-\omega_0^2\dfrac{m}{k}}}.
\end{equation}
Таким образом получаем решение в виде:
\[
\delta x_0\qty(t)=C e^{i\omega_0t}+c.c.
,\] 
\Huge{ОТВЕТ НЕ ЗАВИСИТ ОТ СООТНОШЕНИЯ МАСС, А ФОРМА ОТВЕТА (БОЛЬШОЙ КОРЕНЬ) ЗАВИВИТ ОТ 	m!}
\end{document}
