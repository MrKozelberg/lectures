\documentclass[a4paper,9pt,russian]{article}
\usepackage{cmap}
\usepackage{hyperref}
\usepackage[T2A]{fontenc}
\usepackage[utf8]{inputenc}
\usepackage[russian]{babel}
\usepackage{graphicx}
\usepackage{xcolor}
\usepackage{amssymb}
\usepackage{amsmath}
\usepackage{physics}

\title{Лекция 1}
\author{А. В. Козлов}

\begin{document}
\maketitle
\begin{abstract}
Данная лекция является вводной лекцией курса методов математической физики, читаемой на третьем курсе ВШОПФ Григорием Моисеевичем Жислиным.
\end{abstract}
\section{Классификация линейных дифференциальных уравнений в частных производных второго порядка}
Выдающийся биолог Линей перед созданием своей классификации животных рассмотрел всех известных на тот момент животных.
Так мы поступим и с уравнениями.\par
Как известно из прошлого семестра, уравнение колебаний струны записывается таким образом:
\[
	\pdv[2]{u}{t} = a^2\pdv[2]{u}{x}
.\]
Уравнение теплопроводности:
\[
	\pdv{u}{t}=a^2\pdv[2]{u}{x}
.\] 
Уравнение плоской мембраны под действием стационарных сил:
\[
	\laplacian{u} = f\qty(x,\ y)
.\]
Начнём с самого простого случая.
\subsection{Уравнения с постоянными коэффициентами (случай n переменных)}
Рассмотрим дифференциальное уравнение в частных производных линейное относительно старших коэффициентов с n переменными, где полагаем постоянные коэффициенты перед старшими производными элементами симместричной матрицы:
\begin{equation}
	
\end{equation} 
\end{document}
