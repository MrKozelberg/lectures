\documentclass[a4paper,russian]{article}

\usepackage{cmap}
\usepackage{hyperref}
\usepackage[T2A]{fontenc}
\usepackage[utf8]{inputenc}
\usepackage[russian]{babel}
\usepackage{graphicx}
\usepackage{xcolor}
\usepackage{amssymb}
\usepackage{amsmath}
\usepackage{physics}

\title{Лекция 1}
\author{А. В. Козлов}

\begin{document}
\maketitle
\begin{abstract}
Данная лекция является вводной лекцией курса методов математической физики, читаемой на третьем курсе ВШОПФ Григорием Моисеевичем Жислиным.
\end{abstract}
\section{Классификация линейных дифференциальных уравнений в частных производных второго порядка}
Выдающийся биолог Линей перед созданием классификации животных рассмотрел всех известных на тот момент животных.
Так мы поступим и с дифференциальными уравнениями в частных производных.\par
Как известно из прошлого семестра\cite{enabla_lec}, уравнение колебаний струны записывается таким образом:
\[
	\pdv[2]{u}{t} = a^2\pdv[2]{u}{x}
.\]
Уравнение теплопроводности:
\[
	\pdv{u}{t}=a^2\pdv[2]{u}{x}
.\] 
Уравнение плоской мембраны под действием стационарных сил:
\[
	\laplacian{u} = f\qty(x,\ y)
.\]
Начнём с самого простого случая.
\subsection{Уравнения с постоянными коэффициентами (случай n переменных)}
Рассмотрим дифференциальное уравнение в частных производных линейное относительно старших производных с n переменными, где полагаем постоянные коэффициенты перед старшими производными элементами симместричной матрицы:
\begin{equation}\label{1}
	\sum\limits_{i,j=1}^{n}	A_{ij} \pdv{U}{x_i}{x_j}=\Phi\qty(x_1,\dots,x_n,u_{x_1},\dots,u_{x_n}), \text{ где } A_{ij}=A_{ji}=const
\end{equation}
Зададимся вопросом: какую замену выбрать, чтобы свести рассматриваемое ДУЧП к каноническому виду (то есть к такому виду, что матрица коэффициентов перед старшими прозводными была диагональной, а на диагонали стояили либо $0$, либо $\pm1$)?\par
Сделаем следующую взаимобратную замену (то есть якобиан преобразования не нулевой: $\frac{D\qty(\xi_1,\dots,\xi_n)}{D\qty(x_1,\dots    ,x_n)}\not=0$), где каждый кси есть линейная комбинация иксов:
\begin{equation}
	\xi_s = \sum\limits_{i=1}^{n}\alpha_{si}x_i,\quad s=1,\dots,n.
\end{equation}
Введём следующее обозначение: $V\qty(\xi_1,\dots,\xi_n)=U\qty(x_1,\dots,x_n)$, пересчитаем первую и вторую производные в новых переменных кси:
\begin{align}\label{3}
	\pdv{U}{x_i} &= \sum\limits_{s=1}^{n} \alpha_{si} \pdv{V}{\xi_s}, \\
	\pdv{U}{x_j}{x_i} &= \sum\limits_{t,s=1}^{n} \alpha_{si} \alpha_{tj} \pdv{V}{\xi_t}{\xi_s}.\label{4} 
\end{align}
Подставляем в уравнение (\ref{1}) выражения производных (\ref{3}, \ref{4}):
\begin{equation}\label{5}
	\sum\limits_{s,t=1}^{n}\tilde A_{st}\pdv{V}{\xi_s}{\xi_t}=\tilde \Phi\qty(\xi_1,\dots,\xi_n,V,V_{\xi_1},\dots,V_{\xi_n}),
\end{equation}
где $\tilde A_{st} = \sum\limits_{i,j=1}^{n} \alpha_{si} \alpha_{tj} A_{ij}$. Чтобы понять, как именно привести уравнение к каноническому виду, рассмотрим квадратичную форму $Q$:
\begin{equation}\label{six}
	 Q=\sum\limits_{i,j=1}^{n}A_{ij}x_ix_j
.\end{equation}
И сделаем такое преобразование независимых переменных, что старые переменные выразятся через новые следующим образом:
\[
	x_i = \sum\limits_{s=1}^{n} \beta_{is} \xi_s,\quad x_j = \sum\limits_{t=1}^{n} \beta_{jt} \xi_t
.\] 
Тогда подставляем эти выражения в исходную квадратичную форму (\ref{six}) и получаем:
\begin{equation}\label{7}
	Q = \sum\limits_{s,t=1}^{n} \hat A_{st} \xi_s \xi_t, 
\end{equation}
где $\hat A_{st} = \sum\limits_{i,j=1}^{n} \beta_{is} \beta_{jt} A_{ij}$. Напомним, что до сих пор матрица альф была любой, в связи с чем можно теперь выбрать её такой, что $\norm{\alpha}=\norm{\beta}^T$ или, что тоже самое, $\alpha_{si} = \beta_{is}$. То есть можно выбрать альфа такими, что $\tilde A_{st} = \hat A_{st}$. Тогда матрица $\tilde A$ есть матрица квадратичной формы.\par
Отсюда незамедлительно следует, что её можно диагонализовать в базисе из собственных векторов \cite{en_lec_2}. Перейдя в базис из собственных векторов квадратичной формы $\hat A$, получаем (в этот раз опускаем аргументы функции в правой части):
\begin{equation}\label{diag}
	\sum\limits_{s=1}^{n}\tilde A_{ss} \pdv[2]{V}{\xi_s} = \tilde \Phi(\dots).
\end{equation}
Но сделать это можно огромным множеством способов \cite{en_lec_2}, в зависимости от способа диагонализации могут принимать различное значение диагональные элементы квадратичной формы. Но инварианты квадратичной формы \cite{en_lec_2} будут сохраняться. К ним относят: число зануляющихся диагональных элементов квадртичной формы в диагональном виде $n_0$ и разность количества положительных и отрицательных диагональных элементов квадратичной формы в диагональном виде $\abs{n_+ - n_-}$.
\par
По инвариантам квадратичной формы разумно провести классификацию дифференциальных уравнений в частных производных линейных относительно старших производных с n постоянными переменными. Приведём вариант такой классификации.\par
\paragraph{Типы уравнений}
\begin{enumerate}
	\item $n_- = n$ (то есть все диагональные элементы одного знака). Такой тип уравнений называется {\it эллиптическим}. В качестве примера может быть рассмотрено уравнение плоской мембраны под действием стационарных сил:  $$\laplacian{U}=f(x,\ y).$$
	\item $n_+ = n-1, \  n_- = 1$ (то есть все диагональные элементы единого знака, за исключением лишь одного, который имеет знак противоположный). Такие уравнения называют {\it гиперболическими}. К данному типу уравнений принадлежит уравнение колебаний струны: $$U_{tt} = a^2\laplacian{U} + f(x).$$ 
	\item $n_+ = n-1,\ n_0=1$ (все диагональные элементы единого знака за исключением одного нулевого). Данный тип уравнений называется {\it параболическим}. Проилюстрировать такой тип уравнений можно уравнением теплопроводности: $$U_t = a^2\laplacian{U}+f\qty(x,\ y,\ z).$$
\end{enumerate}\par
Вся эта классификация была проведена для случая с постоянными коэффицентами в (\ref{1}). Рассмотрим случай переменных коэффициентов при старших производных.

\subsection{Классификация уравнений с переменными коэффициентами}
\begin{figure}[h]
\centering
\def\svgwidth{50mm}
   \input{drawing.pdf_tex}	
   \caption{Иллюстрация того, как меняется тип уравнения в зависимости от точки в области $\Omega$.}
\end{figure}
По прежнему работаем с уравнением:
\begin{equation}\label{new_1}
\sum\limits_{i,j=1}^{n} A_{ij}\qty(x_1, x_2, \ldots, x_n) \pdv{U}{x_i}{x_j}=\Phi\qty(x_1,\dots,x_n,u_{x_1},\dots,u_{x_n}),
\end{equation}
где $A_{ij} = A_{ji} \ne const$.\par
Пускай теперь $A_{ij}$ переменная, тогда тип уравнения начинает зависеть от точки. Определим область изменения переменных так: $$\Omega = \qty{x_1,x_2,\ldots,x_n},$$ тогда, чтобы определить тип уравнения в точке $\qty(\bar x_1,\bar x_2,\ldots,\bar x_n)\in\Omega$, нужно посчтитать значение $A_{ij}$ в этой точке, в результате чего получим уравнение с постоянными коэффициентами и, проводя рассуждения, что были ранее проведены для уравнений с постоянными коэффициентами, определяем тип уравнения.  \par
То есть классификация уравнений с переменными коэффциентами такая же, как у уравнений с постоянными коэффициентами, с той лишь оговоркой, что теперь тип уравнения, зависящий от коэффициентов, тоже становится величиной переменной.
\medskip
\subsubsection{Приведение к каноническому виду уравнения второго порядка с переменными коэффициентами}
Рассмотрим уравнение второго порядка. Полагая, что правая часть зависит от всего, кроме вторых производных (откуда следует, что правая часть уравнения не влияет на классификацию), запишем уравнение таким образом:
\begin{equation}\label{n_1}
	A\pdv[2]{U}{x_1}+2B\pdv{U}{x_1}{x_2}+C\pdv[2]{U}{x_2}=\Phi\qty(\ldots).
\end{equation}
Попробуем найти взаимообратное функциональное преобразование, сводящее (\ref{n_1}) к наиболее простому виду. Введём соответсвующие обозначения:
\begin{equation}\label{sys}
\begin{cases}
	\xi=\xi\qty(x_1,x_2)\\
	\eta = \eta\qty(x_1,x_2)
\end{cases}
\text{и } \quad
\frac{D\qty(\xi,\eta)}{D\qty(x_1,x_2)} \ne 0
\end{equation}
Аналогично случаю постоянных коэффициентов вводим обозначение: \[
V\qty(\xi,\ \eta)=U\qty(x_1,\ x_2)
.\] 
Теперь в уравнении (\ref{n_1}) переходим к новым переменным, надеясь, что читатель в силах самостоятельно пересчитать вторые производные в новых переменных, запишем лишь результат. Само уравнение (\ref{n_1}) запишется следующим образом:
\begin{equation}\label{n_2}
	\tilde A\pdv[2]{V}{\xi}+2\tilde B\pdv{V}{\xi}{\eta}+\tilde C\pdv[2]{V}{\eta}=\tilde \Phi\qty(\ldots),
\end{equation}
где коэффициенты в левой части выражаются так (получить самостоятельно):
\begin{equation}\label{coef}
\begin{cases}
	\tilde A = A\qty(\pdv{\xi}{x_1})^2+2B\pdv{\xi}{x_1}\pdv{\xi}{x_2}+C\qty(\pdv{\xi}{x_2})^2\\
	\tilde C = A\qty(\pdv{\eta}{x_1})^2+2B\pdv{\eta}{x_1}\pdv{\eta}{x_2}+C\qty(\pdv{\eta}{x_2})^2\\
	\tilde B = A\pdv{\xi}{x_1}\pdv{\eta}{x_2}+B\qty(\pdv{\xi}{x_1}\pdv{\eta}{x_1}+\pdv{\xi}{x_2}\pdv{\eta}{x_1})+C\pdv{\xi}{x_2}\pdv{\eta}{x_2}
\end{cases}
\end{equation}
Сделаем {\bf важное замечание}. Мы считаем, что уравнение (\ref{n_1}) не вырождается в уравнение первого порядка ни в одной точке, то есть полагаем, что сумма абсолютных значений коэффициентов отлична от нуля $$\abs{A}+\abs{B}+\abs{C} > 0$$ во всей области изменения переменных. 
\par
Докажем следующие утверждение. Если уравнение (\ref{n_1}) не вырождено, то и уравнение (\ref{n_2}) тоже не вырождено. То есть если:
 $$\abs{A}+\abs{B}+\abs{C} > 0,$$ 
то и
 $$\abs{\tilde A}+\abs{\tilde B}+\abs{\tilde C} > 0.$$ 
 \paragraph{Доказательство:}


\newpage
\begin{thebibliography}{5}
\bibitem{enabla_lec}
Лекции Григория Моисеевича Жислина по методам математической физики, доступны, например, здесь:\\ \url{https://enabla.com/ru/set/2/pub/43}
\bibitem{en_lec_2}
Лекции Григория Моисеевича Жислина по линейной алгебре, доступны здесь:\\
\url{https://enabla.com/ru/set/3/pub/65}
\end{thebibliography}

\end{document}
