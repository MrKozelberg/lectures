\documentclass[titlepage]{article}
\usepackage[a4paper,includehead,includefoot,headheight=10pt,headsep=2mm,width=17cm,height=27cm,footskip=0.5cm]{geometry}
\usepackage{cmap}
\usepackage{hyperref}
\usepackage[T1]{fontenc}
\usepackage[utf8]{inputenc}
\usepackage[russian]{babel}
\usepackage{graphicx}
\usepackage{xcolor}
\usepackage{amssymb}
\usepackage{amsmath}
\usepackage{physics}
\usepackage{wrapfig}

\usepackage{pgfplots}
\pgfplotsset{width=7cm,compat=newest}

\usepackage{amsmath}
\DeclareMathOperator\arctanh{arctanh}

\usepackage{amsmath}
\DeclareMathOperator\arccosh{arccosh}

\usepackage{amsmath}
\DeclareMathOperator\const{const}

\usepackage{amsmath}
\DeclareMathOperator\erfi{erfi}

\begin{document}
\begin{flushright}
    Расчётно Графическая Работа №8
    
    Выполнил: Козлов Александр
\end{flushright}
\section{Формулировка}
Найти условие коллапса для решения уравнения:
\begin{equation}
 i\partial_t u + \partial_{xx}u + \partial_{yy}y + \abs{u}^2 u + \alpha x u = -i\gamma u e^{\frac{x}{L}}
\end{equation}
Использовать без аберрационное приближение, считая:
\begin{equation}
 L \gg a_x,\; a_y
\end{equation}
Найти и нарисовать положение центра масс.
\section{Решение}
Восстанавливаем действие по уравнению:
\begin{equation}
 S = \iiint \qty(\dfrac{i}{2} \qty(u^* \partial_t u  -  u\partial_t u^*) - \abs{\partial_x u}^2 - {\abs{\partial_y u}}^2 + \dfrac{{\abs{u}}^4}{2} + \alpha x {\abs{u}}^2) \dd{x} \dd{y} \dd{t}
\end{equation}
Из физических соображений (из-за неимения $y$ в уравнении полагаем, что центр масс не будет смещаться по данной координате, следовательно, не будет $k_y,\; y_0$) выбираем решение в виде следующего гаусса:
\begin{equation}
 u = A \exp\qty[-\dfrac{(x-x_0)^2}{2a_x^2} - \dfrac{y^2}{2a_y^2} + i \qty(\dfrac{\sigma_x (x-x_0)^2}{2} + \dfrac{\sigma_yy^2}{2}+k_xx+\varphi)]
\end{equation}
Подстановка в действие решения такого вида приводит к <<укороченному>> лагранжиану:
\begin{equation}
 \tilde{\mathcal{L}} = -\pi A^2 a_y a_x \qty(\frac14\qty[\dot{\sigma_x}a_x^2+\dot{\sigma_y}a_y^2]+\dot{k_x}x_0+\dot{\varphi}+\frac12\qty[\frac{1}{a_x^2}+\frac{1}{a_y^2}]+\frac12\qty[\sigma_x^2a_x^2+\sigma_y^2a_y^2]+k_x^2-\alpha x_0 - \frac{A^2}{4})
\end{equation}
Тогда, учитывая диссипативную функцию, получаем следующие уравнения Лагранжа:
\begin{equation}
 \pdv{\tilde{\mathcal{L}}}{a_j} - \dv{t}(\pdv{\tilde{\mathcal{L}}}{\dot{a_j}}) = -i\gamma\iint\dd{x}\dd{y}e^{\frac{x}{L}}\qty(u \pdv{u^*}{a_j}-u^* \pdv{u}{a_j}),\qquad j=\overline{1,8}
\end{equation}
Для $a_j = \varphi$ получается уравнение:
\begin{equation}\label{P}
 \dot{\mathrm{P}} = -2\gamma \mathrm{P} e^{\frac{x_0}{L}}, \qquad \mathrm{P} =\pi A^2a_xa_y
\end{equation}
Для $a_j = \sigma_x$:
\begin{equation}
 \sigma_x = \dfrac{\dot{a_x}}{2a_x}
\end{equation}
Для $a_j = a_x^2$:
\begin{equation}
 \frac{\dot{\sigma_x}}{2} + \sigma_x^2 = \frac{1}{a_x^4} \qty(1 - \frac{\mathrm{P}a_x}{4\pi a_y})
\end{equation}
Откуда, подставив $\sigma_x$, можно получить:
\begin{equation}
 \ddot{a_x} = \frac{4}{a_x^3} \qty(1 - \frac{\mathrm{P}a_x}{4\pi a_y})
\end{equation}
Аналогично и для игрековых вариантов этих переменных:
\begin{equation}
 \ddot{a_y} = \frac{4}{a_y^3} \qty(1 - \frac{\mathrm{P}a_y}{4\pi a_x})
\end{equation}
Варьирование по $A^2$ может позволить найти $\varphi$, если известны $a_x,\; a_y$, ведь оно приводит к соотношению:
\begin{equation}
 \mathrm{P} = 2a_xa_y\qty(\frac14\qty[\dot{\sigma_x}a_x^2+\dot{\sigma_y}a_y^2]+\dot{\varphi}+\frac12\qty[\frac{1}{a_x^2}+\frac{1}{a_y^2}]+\frac12\qty[\sigma_x^2a_x^2+\sigma_y^2a_y^2]+k_x^2)
\end{equation}
Варьирование по $x_0$ приводит к следующему соотношению:
\begin{equation}
 \dot{k_x} = \alpha
\end{equation}
Откуда можно получить:
\begin{equation}
 k_x\qty(t) = \alpha t + k_0
\end{equation}
Варьирование по $k_x$ даёт:
\begin{equation}
 k_x = \dfrac{\dot{x_0}}{2}
\end{equation}
\begin{figure}[t]
\centering
 \begin{tikzpicture}
	\begin{axis}
	[
	xlabel={$t$},
	ylabel={$x_0$},  
	axis lines=middle,
	enlargelimits=true,
	xmin=-0.5,
	xmax=5,
	restrict y to domain=0:5
    ]
		\addplot[red!50,domain={0:5},samples=1000]{1.5*x*x/2+1*x+0.89} node[pos=1] (1) {};
		\addplot[blue!50,domain={0:5},samples=1000]{1*x+0.89} node[pos=1] (2) {};
		\addplot[brown!50,domain={0:5},samples=1000]{-1.5*x*x/2+1*x+0.89} node[pos=1] (3) {};
		\node [above, color = red] at (1) {$\alpha = 1.5$};
		\node [above, color = blue] at (2) {$\alpha = 0$};
		\node [above, color = brown] at (3) {$\alpha = -1.5$};
	\end{axis}
\end{tikzpicture}
\caption{Траектория центра масс с различными параметрами $\alpha$ при параметрах $k_0 = 0.5$ и $\tilde{x}_0 = 0.89$}
\label{pic:speed}
\end{figure}
Что позволяет выяснить траекторию центра масс решения:
\begin{equation}
 x_0\qty(t) = \alpha \dfrac{t^2}{2} + 2 k_0 t +\tilde{x}_0
\end{equation}
Данный результат при подстановке в уравнение (\ref{P}) позволяет найти:
\begin{equation}
 \mathrm{P} \qty(t) = \mathrm{P_0} \exp{-2\gamma e^{\frac{\tilde{x}}{2} - \frac{k_0^2}{2\alpha}} \sqrt{\frac{\pi}{2\alpha}}\erfi{\qty[\frac{\alpha t + k_0}{\sqrt{2\alpha}}]}}
\end{equation}
Это функция очень сильно затухает. Будем считать далее инкремент затухания достаточно большим:
\begin{equation}
\gamma \gg e^{-\frac{\tilde{x}}{2} + \frac{k_0^2}{2\alpha}} \sqrt{\dfrac{2a}{\pi}} \dfrac{1}{\erfi{\qty[\frac{\alpha + k_0}{\sqrt{2\alpha}}]}} 
\end{equation}
Таким образом система уравнений на $a_x,\; a_y$ оказывается замкнутой:
\begin{align}
 \begin{cases}
 &\ddot{a_x} = \dfrac{4}{a_x^3} \qty(1 - \dfrac{\mathrm{P} a_x}{4\pi a_y})\simeq \dfrac{4}{a_x^3}\\
 &\ddot{a_y} = \dfrac{4}{a_y^3} \qty(1 - \dfrac{\mathrm{P}a_y}{4\pi a_x}) \simeq \dfrac{4}{a_y^3}
 \end{cases}
\end{align}
То есть при больших гамма коллапса не будет. Посмотрим, что будет при малых гамма:
\begin{equation}
\gamma \ll e^{-\frac{\tilde{x}}{2} + \frac{k_0^2}{2\alpha}} \sqrt{\dfrac{2a}{\pi}} \dfrac{1}{\erfi{\qty[\frac{\alpha + k_0}{\sqrt{2\alpha}}]}} 
\end{equation}
Тут можно действовать следующим образом. Аппроксимировать $\mathrm{P} \simeq \mathrm{P_0}$ до какого-то времени (см. график $\mathrm{P}\qty(t)$). Таким временем будет характерное время $t_*$ такое, что:
\begin{equation}
 2\gamma e^{\frac{\tilde{x}}{2} - \frac{k_0^2}{2\alpha}} \sqrt{\frac{\pi}{2\alpha}}\erfi{\qty[\frac{\alpha t_* + k_0}{\sqrt{2\alpha}}]} = \dfrac{1}{e}
\end{equation}
То есть это характерная ширина $\mathrm{P}\qty(t)$. Получается следующее приближение:
\begin{gather}
 \mathrm{P}\qty(t) \simeq
 \begin{cases}
  &\mathrm{P_0},\quad {t} \leq t_*\\
  &0,\quad {t} > t_*
 \end{cases} 
\end{gather}
Тогда систему при $t < t_*$ можно упростить:
\begin{align}
 \begin{cases}
 &\ddot{a_x} = \dfrac{4}{a_x^3} \qty(1 - \dfrac{\mathrm{P_0}}{4\pi}\dfrac{a_x}{a_y})\\
 &\ddot{a_y} = \dfrac{4}{a_y^3} \qty(1 - \dfrac{\mathrm{P_0}}{4\pi} \dfrac{a_y}{a_x})
 \end{cases}
\end{align}
Коллапс может быть лишь при $t < t_*$ ввиду того, что при больших временах слева в уравнениях появляется положительный коэффициент, и ширины будут лишь нарастать.
\par Легко проанализировать случай, когда начальные значения $a_x,\; a_y$ близки. Из симметрии уравнений следует, что близость $a_x,\; a_y$ сохраняется в дальнейшем (это так же видно из численного решения). Систему можно привести к более простому виду:
\begin{align}
 \begin{cases}
 &\ddot{a_x} = \dfrac{4}{a_x^3} \qty(1 - \dfrac{\mathrm{P_0}}{4\pi})\\
 &\ddot{a_y} = \dfrac{4}{a_y^3} \qty(1 - \dfrac{\mathrm{P_0}}{4\pi} )
 \end{cases}
\end{align}
Схожие уравнения уже были проанализированы на лекции. Условием существования коллапса будет:
\begin{equation}
\mathrm{P_0} > 4 \pi 
\end{equation}
%
%Замечание:
%Альфа равно нулю следовательно нет каустики, альфа больше нуля --- каустика есть
%


\end{document}

