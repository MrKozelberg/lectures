% document type
\documentclass[12pt]{article}

% packages
\usepackage[total={170mm,230mm}]{geometry}
\usepackage[utf8]{inputenc}
\usepackage[T2A]{fontenc}
\usepackage[russian]{babel}
\usepackage{graphicx}
\usepackage{xcolor}
\usepackage{amssymb}
\usepackage{amsfonts}
\usepackage{amsmath}
\usepackage{amsthm}
\usepackage{physics}
\usepackage{nicefrac}
\usepackage{wrapfig}
\usepackage{cancel}
\usepackage{hyperref, cmap}
\usepackage{pgfplots}

% settings
\pgfplotsset{width=\linewidth, compat=newest}

% definitions
\DeclareMathOperator\arctanh{arctanh}
\DeclareMathOperator\arccosh{arccosh}
\DeclareMathOperator\const{const}
\newtheorem{definition}{Опредление}[section]
\newtheorem{theorem}{Теорема}[section]
\newtheorem{axiom}{Аксиома}[section]
\newtheorem{hypothesis}{Гипотеза}[section]

\title{Число e и экспоненциальная зависимость}
\author{Алексей Савватеев \and Александр Тонис}

\begin{document}
\maketitle
\begin{abstract}
В данной лекции рассмотрены комбинаторные и вероятностные математические задачи, приводящие к числу $e$. Введено понятие экспоненциальной зависимости.
\par
Конспектировал Александр Козлов. 
\end{abstract}
\newpage
\tableofcontents
\newpage

% 00:00:00
В настоящей лекции будет введено понятие об экспоненте и экспоненциальной зависимости. Сперва мы рассмотрим задачу с забавной формулировкой. Задача приведёт нас к числу $e$ как таковому, затем мы будем вводить функциональную зависимость $e^x$, а так же прочие степенные зависимости. Все выкладки и строгие математические доказательства будут делаться по ходу изложения.

\section{Задача про пьяницу} % 00:01:15

\subsection{Формулировка} % 00:01:15
\label{sub:f}
Пускай пьяница идёт домой. Ему остаётся ровно $1\ \text{м}$ до ворот его дома. Каждый последующий шаг нетрезвого человека случаен, длина шага $l$ (измеряется в метрах) удовлетворяет следующему двойному неравенству:

\begin{equation}
	0 < l \le 1.
\end{equation}
Сколько шагов в среднем совершит пьяница, перед тем, как он окажется дома?

\subsubsection{Замечание} % 00:04:05
Стоит обратить внимание на то, что формулировка задачи позволяет существовать такой не отвечающей реальности сутации, что пьяница не сможет за конечное время преодолеть расстояние до своего дома. Например, такая ситуации случится, если пьяный человек пойдёт с длиной шага $l_n = {1}/{n^2}$.

\subsection{Решение} % 00:06:11
Зададимся вопросом, что значат слова "{}число шагов"{}. Такой парой слов обозначается натуральное число, равное количеству шагов, которое пьяницы должен сделать, чтобы оказаться дома. Дома он окажется тогда, когда 

\begin{equation}
	l_1 + l_2 + \ldots + l_n \ge 1,
\end{equation}
где за $l_i$ обозначена длина $i$\--ого шага.

\par Посчитаем, чему равны вероятности попадания пьяницы домой за различное число шагов. Так как длина каждого шага является не зависящей ни от чего случайной величиной, то такой вариант, что пьяница шагнёт прямо в ворота своего дома, то есть случай 

\begin{equation}
	l_1 + l_2 + \ldots + l_n = 1,
\end{equation}
здесь и далее отбросим из рассмотрения. Таким образом, ясно, что за один шаг пьяный человек не способен предолеть расстояние в $1\ \text{м}$. Данный вывод запишется как $p_1 = 0$, где за $p_1$ обозначена вероятность попадания пьяницы домой за один ход. Вероятность попадания за один ход будет вероятностью того, что 

\begin{equation}
	l_1 + l_2 > 1,
\end{equation}
и будет обозначаться аналогично предыдущей вероятности через $p_2$. Для трёх шагов имеем вероятностью попадания пьяницы домой символ $p_3$, за который обозначается вероятность того, что 

\begin{equation}\label{eq:old}
	\begin{cases}
		& l_1 + l_2 + l_3 > 1,\\
		& l_1 + l_2 \le 1.
	\end{cases}
\end{equation}

Можно продолжать и дальше. Для каждого числа шагов $n$ будем обозначать вероятность попадания домой за $p_n$. И все данные вероятности, кроме самой первой, будут ненулевыми.

\par Вспомним, что нам требуется найти среднее значение количества шагов. Среднее в теории вероятности называется \emph{математическим ожиданием}. Из теории вероятности известно (см. \cite{w1}), что для случайной дискретной величины $X$ с \emph{распределением} (то есть с вероятностью случиться каждого из элементов множетсва $X$)

\begin{equation}
	\qty{p_i}_{i=1}^\infty
\end{equation}
таким, что 

\begin{equation} \label{eq:norm_old}
	\sum_{i=1}^\infty{p_i} = 1,
\end{equation}
математическое ожидание вычисляется по следующей формуле 

\begin{equation}
	M\qty[X] = \sum_{i=1}^\infty{x_i \cdot p_i}, 
\end{equation}
где через $M$ обозначено математическое ожидание. Тогда для ответа на вопрос о среднем значении количества шагов, потребуется узнать, чему равно выражение 

\begin{equation} \label{eq:sum_old}
	1\cdot p_1 + 2\cdot p_2 + \ldots + n\cdot p_n + \ldots
\end{equation}
Но может ли вообще данная сумма иметь конечное значение? Докажем позже, что такой ряд сходится. 

\par А пока зададимся вопросом о том, как сосчитать эту сумму. Для этого перейдём к другому принципу вычисления вероятностей. Теперь будем рассуждать в терминах вероятностей 

\begin{equation}
	p_{\ge1},\ p_{\ge2},\ \ldots,\ p_{\ge n},\ \ldots
\end{equation}
То есть теперь будем оперировать вероятностями $p_{\ge n}$, которые обозначают вероятность того, что пьянице потребуется не меньше $n$ шагов, чтобы добраться до своего жилища. Иными словами $p_{\ge n}$ отражает вероятность того, что 

\begin{equation} 
	l_1 + l_2 + \ldots + l_{n-1} \le 1.\label{eq:new}
\end{equation}
Переходим мы к таким вероятностям ради удобства их вычисления, ведь теперь требуется выполнение одного условия (\ref{eq:new}) вместо двух условий (\ref{eq:old}). Связь между вероятностями записывается очевидным образом

\begin{equation}
	p_n = p_{\ge n} - p_{\ge n+1}.
\end{equation}
Теперь можно переписать сумму (\ref{eq:sum_old}) через новые вероятности. Если привести подобные, то можно получить

\begin{equation} \label{eq:sum_new}
\begin{split}
	&1\cdot (p_{\ge 1} - p_{\ge 2}) + 2\cdot (p_{\ge 2} - p_{\ge 3}) + \ldots + n\cdot (p_{\ge n} - p_{\ge n+1}) + \ldots \\
	& = p_{\ge 1} + p_{\ge 2} + p_{\ge 3} + p_{\ge 4} + \ldots
\end{split}
\end{equation}
Справедливость приведения подобных в бесконечной сумме оставим на потом, полагая приведение подобных в данном примере справедливым. Таким образом, мы свели задачу о среднем количестве шагов пьяницы к выражению (\ref{eq:sum_new}).

% 00:16:37
\par Заметим, что новые вероятности не являются вероятностями взаимоисключающих событий. Теперь сумма (или нормировка) всех вероятностей уже не будет равна единице, так как суммируются вероятности взаимоперекрывающихся событий. То есть событие "{}не меньше пяти шагов"{} включает в себя событие "{}не меньше шести шагов"{}, например.

\par Первые два слагаемых в выражении (\ref{eq:sum_new}) равны единицам

\begin{equation}
	p_{\ge 1} = p_{\ge 2} = 1,
\end{equation}
так как, чтобы достичь ворот дома, пьяница должен сделать как минимум два шага, как было показано выше (то есть пьяный человек всегда делает хотя бы два шага).

\par Вычислим вероятности остальных событий. Начнём с $p_{\ge 3}$. Данная вероятность является вероятностью события, когда пьяница сделал не меньше трёх шагов, что равноценно событию, когда пьяница сделал два шага и не достиг ворот дома. То есть "{}не меньше трёх шагов"{} эквивалентно "{}двух шагов мало"{}. Вычислим вероятность события 

\begin{equation} \label{uneq:1}
	l_1 + l_2 \le 1.
\end{equation}
То, что шаги независимы, означает, что нужно отлкадывать две координатные оси, перпендикулярные друг другу, которые отвечают длинам шагов. Тогда получим квадрат со стороной $1\ \text{м}$, заполненный точками с координатами $\qty(l_1,\ l_2)$. Все события (то есть каждая пара длин шагов) равновероятны, поэтому, чтобы определить вероятность события, когда двух шагов мало для достижения ворот дома, нужно определить площадь области, где выполняется неравенство (\ref{uneq:1}). Данная область закрашена зелёным на рисунке \ref{fig:figure1}.

\begin{figure}[htbp]
	\centering
	\begin{tikzpicture}
	\begin{axis}[
		axis lines=middle,
		ymax = 1.05,
		ymin = -0.05,
		xmax = 1.05,
		xmin = -0.05,
		line width=.1cm,
		%enlargelimits=true,
		%restrict y to domain=-200:200,
		xtick=\empty,
		ytick=\empty
		]
		\draw (0,0) -- (1,0) -- (1,1) -- (0,1) -- cycle;
		\draw[fill=green] (0,0) -- (1,0) -- (0,1) -- cycle;
	\end{axis}
	\node at (1.3,13) {\Large $l_2$};
	\node at (15.2,1.2) {\Large $l_1$};
	\draw[line width = 0.1cm] (1.25,12.49) -- (0,12.49) node[black!, left] {\Large $1$};
	\draw[line width = 0.1cm] (1.25,0.6) -- (0,0.6) node[black!, below right] {\Large $0$};
	\draw[line width = 0.1cm] (14.72,1) -- (14.72,0) node[black!, anchor=north] {\Large $1$};
	\end{tikzpicture}
	\caption{Иллюстрация к вычислению вероятности $p_{\ge 3}$. В осях $l_1$, $l_2$ построен квадрат, отражающий все возможные варианты двух шагов. Зелёной области соответсвуют такие точки, что каждой из них отвечает случай, когда пьяница не дошёл до ворот за два шага.}
	\label{fig:figure1}
\end{figure}
Площадь зелёной области, очевидно, равна $1/2$. Тогда

\begin{equation}
	p_{\ge 3} = \dfrac{1}{2}.
\end{equation}

\par Аналогичные рассуждения проводим для $p_{\ge 4}$. На этот раз строим куб со стороной $1\ \text{м}$ и ищем объём фигуры, которая лежим ниже плоскости 

\begin{equation}
	l_3 = -l_1 -l_2 +1.
\end{equation}
Это будет пирмамида с площадью основания $1/2$ и высотой $1$. Тогда 

\begin{equation}
	p_{\ge 4} = \dfrac{1}{6}.
\end{equation}
Перейдём сразу к случаю $n$ измерений. Фигуру, образованную в координатном пространстве $(l_1,\ l_2,\ \ldots, l_n)$ точками, удовлетворяющими условиям

\begin{equation}
	\begin{split}
		& 0 \le l_i \ge 1,\ \text{где}\ i=1,\ \ldots, n,\\
		& l_1 + l_2 + \ldots + l_n \ge 1,
	\end{split}
\end{equation}
называют \emph{n~\--~мерным симплексом}. Оказывается (см. \cite{w2}), что объёмы соседних по размерности симплексов связаны соотношением

\begin{equation}
	V_{n+1} = \dfrac{1}{n} V_n.
\end{equation}
Так как нами уже были вычислены объёмы симплексов малой размерности, то можно получить методом математической индукции следующую формулу для объёма симплекса:

\begin{equation}
	V_n = \dfrac{1}{n!}.
\end{equation}
Тогда получаем такой ответ для среднего числа шагов пьяницы

\begin{equation}\label{eq:sum1}
	1 + 1 + \dfrac12 + \dfrac16 + \ldots + \dfrac{1}{n!} + \ldots
\end{equation}
Утверждается, что данный ряд сходится (см. \cite{sum}). Напомним, почему это верно. Чтобы доказать, что ряд сходится, оценим его сверху рядом, который заведомо сходится. Так как для $n > 1$

\begin{equation}
	\dfrac{1}{n!} \le \dfrac{1}{n \cdot (n-1)},
\end{equation}
то сумму (\ref{eq:sum1}) оцениваем сверxу суммой, которую уже считали ранее (см. \cite{sum})

\begin{equation}
\begin{split}
	&1 + 1 + \dfrac{1}{1\cdot 2} + \dfrac{1}{2\cdot 3} + \dfrac{1}{3\cdot 4} + \ldots \\
	&= 1 + 1 + \qty(1 - \dfrac12) + \qty(\dfrac12 - \dfrac13) + \qty(\dfrac13 - \dfrac14) + \ldots\\
	&=3
\end{split}
\end{equation}
Таким образом, сумма ряда (\ref{eq:sum1}) меньше трёх. Введём важное определение математического анализа.

\begin{definition}
	Числом Эйлера (или числом $e$) называют сумму ряда

	\begin{equation}
	1 + 1 + \dfrac12 + \dfrac16 + \ldots + \dfrac{1}{n!} + \ldots
	\end{equation}
\end{definition}

\section{Многочлены, ряды и элементарные функции} % 00:28:27
\subsection{Многочлены}
Начнём с рассмотрения многочленов (см. \ref{fig:71}). Независимая переменная $x$ входит многочлены лишь с помощью операций сложения, умножения и вычитания, кроме того, привлекаются вещественные числа. 

\begin{figure}[htbp]
\centering
\begin{tikzpicture}
\begin{axis}[
xlabel={\Large $x$},
ylabel={\Large $f(x)$},
axis lines=middle,
enlargelimits=true,
%restrict y to domain=-200:200,
xtick=\empty,
ytick=\empty,
line width=0.1cm,
]
\addplot[
blue!,
line width=0.1cm,
domain={-2:2},
samples=100
]{-x^3 + x + 0.25};
\end{axis}
\end{tikzpicture}
\caption{График параболы $f(x) = -x^3 + x + 0.25$ изображён синей кривой.}
\label{fig:71}
\end{figure}
Данное соображение можно записать в следующем виде:

\begin{equation}
	\qty{\mathbb{R}; x; +,\ -, \cdot} \Longrightarrow a_0 + a_1 \cdot x + a_2 \cdot x^2 + \ldots + a_n \cdot x^n.
\end{equation}
Многочлены можно складывать, вычитать и умножать. Данные операции не выводят за пределы множества многочленов. 

\subsection{Экспонента}
Если решать физическую задачу о том, как с высотой меняется давление воздуха в атмосфере, то для описания результата полиномиальных функций уже не хватит. Тут пригождается экспонента. Существует несколько путей получаения экспоненты.

\par К экспоненте можно прийти, рассматривая бесконечный многочлен (ряд). Пока опуская подробности вывода, запишем в некотором роде определение экспоненты

\begin{equation}
	e^x = \exp(x) = 1 + x + \dfrac{x^2}{2!} + \dfrac{x^3}{3!} + \ldots
\end{equation}

\par С другой стороны можно рассматривать функцию двух переменных $y^x$ и анализируя при каких значениях аргементов функция может существовать, а при каких~\----~не может, тоже получим экспоненту.

\section{Задача о доле перестановок} %00:38:28
\subsection{Формулировка} %00:38:35
Изучим ещё один пример, из которого возникает число $e$. Будем рассматривать перестановки некоторого множества. Например, множества 

\begin{equation}
 	N = \qty{1,\ 2, \ \ldots,\ n}.
 \end{equation} 
Перестановка этого множества является взаимооднозначным отображением $f$

\begin{equation}
	f:\ N\longrightarrow N.
\end{equation}
Зададимся вопросом о том, какое количество таких перестановок не обладают неподвижной точкой. Напомним, определение неподвижной точки.

\begin{definition}
	Точка $k$ называется неподвижной точкой отображения $f$, если 

	\begin{equation}
		f(k) = k.
	\end{equation}
\end{definition}

\subsection{Решение} %00:39:58
Чтобы посчитать количество требуемых перестановок, рассмотрим множество всех перестановок, всего их $n!$, и выделим множество перестановок, которые оставляют на месте число $1$. То есть число $1$ будет фиксировано, а остальные числа могут быть любыми. Тогда пускай на первом месте стоит число  $1$, на второе место будет вставать $(n-1)$ чисел, на третье уже $(n-2)$ и так далее. Тогда, перемножив все возможные варианты постановок в наборе, получим 

\begin{equation}
	1 \cdot (n-1) \cdot (n-2) \cdot \ldots \cdot (n-(n-1)) = (n-1)!
\end{equation}

Столько существует перестановок при фиксированном числе $1$. Ясно, что аналогичный ответ получится для перестановок при любом другом одном фиксированном числе. Мы хотим выяснить число перестановок, не обладающих неподвижной точкой. Поэтому, казалось бы, нужно из общего числа перестановок $n!$ вычесть все перестановки с любым одним фиксированным числом. Чисел у нас $n$, число перестановок при любом одном фиксированном числе равна $(n-1)!$. Значит, приходим к выражению

\begin{equation} \label{eq:s0}
	n! - n\cdot(n-1)!
\end{equation}
Данное выражение было бы ответом, если бы мы не подсчитывали два раза перестановки, в которых фиксированны два любых числа. Так, например, для $n=2$ мы бы учли два раза перестаковки при фиксированной $1$ и при фиксированной $2$ (см. рис. \ref{fig:figure11}). 

\begin{figure}[htbp]
	\centering
	\begin{tikzpicture}
		\draw[line width=0.1cm] (-1,0) ellipse (2 and 1); 
		\draw[line width=0.1cm] (1,0) ellipse  (2 and 1);
		\node at (0,0) {\Large $++$};
		\node at (-1.5,0) {\Large $+$};
		\node at (1.5,0) {\Large $+$};
		\node at (-2.3,-1.2) {\Large $A_1$};
		\node at (2.3,-1.2) {\Large $A_2$};
	\end{tikzpicture}
	\caption{Диаграмма Эйлера, иллюстрирующая подсчёт перестановок для $n=2$. Множество $A_1$ является множеством перестановок с фиксированной единицей, а множество $A_2$~\----~с фиксированной двойкой. Количество плюсов в каждой области диаграммы показывает, сколько раз были учтены данные области.}
	\label{fig:figure11}
\end{figure}
Значит, стоит вычесть лишние учтённые перестановки. Посчитаем их количество. Для этого вычислим количество перестановок с фиксированными двумя любыми числами. Оно будет равно

\begin{equation}
	1 \cdot 1 \cdot (n-2) \cdot (n-3) \cdot \ldots \cdot (n - (n-2)) = (n-2)!
\end{equation}
Чтобы учесть все возможные пары фиксированных чисел, данную величину следует умножить на количество таких пар. Их число равно числу сочетаний из $n$ по $2$

\begin{equation}
	C_n^2 = \dfrac{n!}{2! \cdot (n-2)!} = \dfrac{n (n-1)}{2}
\end{equation}
Тогда выражение (\ref{eq:s0}) уточняется и приводится к виду 

\begin{equation} \label{eq:s1}
	n! - \qty(n\cdot(n-1)! - \dfrac{n (n-1)}{2}\cdot (n-2)! ).
\end{equation}
Но и это не ответ. В таком подсчёте мы совершенно выбросили из рассмотрения перестановки, где фиксированны три числа. Действительно, обратимся к диаграммам Эйлера (см. рис. \ref{fig:figure2}).

\begin{figure}[htbp]
	\centering
	\begin{tikzpicture}
		\draw[line width=0.1cm] (-1,0) circle (2); 
		\draw[line width=0.1cm] (1,0) circle  (2);
		\draw[line width=0.1cm] (0,1.4) circle  (2);
		\node at (0,0.3) {\Large $+++$};
		\node at (0,-0.2) {\Large $---$};
		\node at (-1.3,1.3) {\Large $++$};
		\node at (-1.3,0.8) {\Large $-$};
		\node at (1.3,1.3) {\Large $++$};
		\node at (1.3,0.8) {\Large $-$};
		\node at (0,-1.4) {\Large $-$};
		\node at (0,-0.9) {\Large $++$};
		\node at (-1.4,-1) {\Large $+$};
		\node at (1.4,-1) {\Large $+$};
		\node at (0,2.5) {\Large $+$};
		\node at (-3,-1.2) {\Large $A_1$};
		\node at (3,-1.2) {\Large $A_2$};
		\node at (1.5,3.4) {\Large $A_3$};
	\end{tikzpicture}
	\caption{Диаграмма Эйлера, иллюстрирующая подсчёт перестановок для $n=3$. Множество $A_1$ является множеством перестановок с фиксированной единицей, а множество $A_2$~\----~с фиксированной двойкой и множество $A_3$~\----~с фиксированной тройкой. Количество плюсов и минусов в каждой области диаграммы показывает, сколько раз были учтены данные области.}
	\label{fig:figure2}
\end{figure}
Видно, что пересечение трех множество не учитывается при подсчёте перестановок. В этот раз мы не досчитываемся перестановок. Поэтому требуется прибавить недостающие перестановки. Аналогично случаю с двойкой получаем, что недостающее число равно

\begin{equation}
	C_n^3 \cdot (n-3)! = \dfrac{n (n-1) (n-2)}{3!} \cdot (n-3)!
\end{equation}
Тогда более точная формула для числа перестановок запишется следующим образом:

\begin{equation}
	n! - \qty(n\cdot(n-1)! - \dfrac{n (n-1)}{2}\cdot (n-2)! + \dfrac{n (n-1) (n-2)}{3!} \cdot (n-3)!).
\end{equation}
Ясно, что теперь у нас возникают проблемы с перестановками, где фиксированны сразу четыре числа, (их  будет слишком много) и так далее. То есть по сути ответ на вопрос задачи запишется в виде знакочередующейся суммы

\begin{equation}
	Q_n = n! - n\cdot(n-1)! + \dfrac{n (n-1)}{2}\cdot (n-2)! - \dfrac{n (n-1) (n-2)}{3!} \cdot (n-3)! + \ldots
\end{equation}
Каждый член данной суммы равен $n!$, домноженному на некоторый фактор. Вынесем общий множитель за скобку

\begin{equation}
	Q_n = n!\qty(\dfrac{1}{0!} - \dfrac{1}{1!} + \dfrac{1}{2!} - \dfrac{1}{3!} + \dfrac{1}{4!} + \ldots + \dfrac{(-1)^n}{n!}).
\end{equation}
Можно заметить, что сумма в скобках является частичной суммой ряда для экспоненты 

\begin{equation}
	\exp(x) = e^x = 1 + x + \dfrac{x^2}{2!} + \dfrac{x^3}{3!} + \ldots
\end{equation}
при $x=-1$. Тогда рассмотрим предел 

\begin{equation}
	\lim_{n\rightarrow \infty} {\dfrac{Q_n}{n!}} = \exp(-1) \approx 0.37.
\end{equation}
Таким образом, если мы возьмём любую перестановку, то она с вероятностью $37\%$ не будет иметь ни одной неподвижной точки.

\begin{thebibliography}{9}
	\bibitem{w1}
	Wikipedia, \textit{Математическое ожидание}
	\\\href{https://ru.wikipedia.org/wiki/\%D0\%9C\%D0\%B0\%D1\%82\%D0\%B5\%D0\%BC\%D0\%B0\%D1\%82\%D0\%B8\%D1\%87\%D0\%B5\%D1\%81\%D0\%BA\%D0\%BE\%D0\%B5\_\%D0\%BE\%D0\%B6\%D0\%B8\%D0\%B4\%D0\%B0\%D0\%BD\%D0\%B8\%D0\%B5}{https://ru.wikipedia.org/wiki/Математическое\_ожидание}

	\bibitem{w2}
	Wikipedia, \textit{Симплекс}
	\\\href{https://ru.wikipedia.org/wiki/\%D0\%A1\%D0\%B8\%D0\%BC\%D0\%BF\%D0\%BB\%D0\%B5\%D0\%BA\%D1\%81}{https://ru.wikipedia.org/wiki/Симплекс}

	\bibitem{sum}
	Савватеев А., Тонис А., \textit{Ряды с неотрицательными слагаемыми}
	\\\texttt{\url{https://enabla.com/pub/613}}
\end{thebibliography}
\end{document}