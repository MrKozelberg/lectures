% document type
\documentclass[12pt]{article}

% packages
\usepackage[total={170mm,230mm}]{geometry}
\usepackage[utf8]{inputenc}
\usepackage[T2A]{fontenc}
\usepackage[russian]{babel}
\usepackage{graphicx}
\usepackage{xcolor}
\usepackage{amssymb}
\usepackage{amsfonts}
\usepackage{amsmath}
\usepackage{amsthm}
\usepackage{physics}
\usepackage{nicefrac}
\usepackage{wrapfig}
\usepackage{cancel}
\usepackage{hyperref, cmap}
\usepackage{pgfplots}

% settings
\pgfplotsset{width=\linewidth, compat=newest}

% definitions
\DeclareMathOperator\arctanh{arctanh}
\DeclareMathOperator\arccosh{arccosh}
\DeclareMathOperator\const{const}
\newtheorem{definition}{Опредление}[section]
\newtheorem{theorem}{Теорема}[section]
\newtheorem{axiom}{Аксиома}[section]
\newtheorem{hypothesis}{Гипотеза}[section]

\title{Лекция 6}
\author{Алексей Савватеев \and Александр Тонис}

\begin{document}
\maketitle
\begin{abstract}
В данной лекции
\par
Конспектировал Александр Козлов. 
\end{abstract}
\newpage
\tableofcontents
\newpage
В настоящей лекции будет введено понятие об экспоненте и экспоненциальной зависимости. Сперва мы рассмотрим некоторую задачу с забавной формулировкой. Задача приведёт нас к числу $e$ как таковому, затем мы будем вводить функциональную зависимость

\begin{equation}
	e^x.
\end{equation}

Все выкладки и строгие математические доказательства будут делаться по ходу изложения.

\section{Задача про пьяницу}

Сегодня мы будем вводить экспоненциальную зависимость. Сначала рассмотрим весёлую задачу, которая приводит к числу Эйлера ($e$) как таковому, затем б


\end{document}
